Our experimental results were still interesting despite some of the issues
we ran into. Often, running the \verb|zmap|~\ref{zmap} scan would cause our
home router to shutdown during long scans, so to make up for that we did
several shorter scans to increase our sample size, keeping only unique IPs.
We also approached scanning by creating a tool to establish connections to
each individual IP, but even with multithreading it could not reach the speed
of \verb|zmap|~\ref{zmap}. In the end, we decided to include that method in
the pipeline as an additional filter to make sure we have a list of IPs
that we can create connections to.

For CVE-2014-2256, the Shodan search using the term \\\verb|siemens port:"102"| returned
1,499 results, with the top five countries the results are from being
Germany, Italy, United States, Spain, and United Kingdom, as shown in
Figure~\ref{fig:shodan-siemens}. Our custom tool, on the other hand,
discovered 2957 internet facing devices on port 102 after a few zmap scans
of 10000 seconds each. Then, after grabbing banners for each IP and filtering
the banner information, we see that 2500 are open and 59 are vulnerable. Compared
to the Shodan search, this is not a lot of results, but shows that even Shodan
is not necessarily needed to find this vulnerability in a device. A direct
comparison between the number of results found for both Shodan and our custom
tool, along with ratios of vulnerable devices detected to devices probed, is
given in Table~\ref{tab:cve-2014}.

\begin{figure}[!h]
\begin{center}
\psfig{figure=images/port-102.eps,width=2.5in,height=3.75in} 
\caption{Shodan search results for CVE-2014-2256}
\label{fig:shodan-siemens}
\end{center}
\end{figure}

\begin{table}[!h]
\begin{tabular}{l|l|l}
                             & Shodan & Custom Tool \\ \hline
Scan on Port 102             & 21,843 & 2,500       \\ \hline
Vulnerable Devices Detected  & 1,401  & 59          \\ \hline
Vulnerable to Detected Ratio & 0.0641 & 0.0236      \\
\end{tabular}
\caption{Comparison of Shodan results and our custom tool's results for CVE-2014-2256.}
\label{tab:cve-2014}
\end{table}

For CVE-2019-0708, the Shodan search for \verb|port:"3389"| returned
4,977,976 results, but would need to be narrowed down based on the operating
system. Adding the term \verb|os:"Windows 7 or 8"| to the original search term returned
14,012 results, shown in Figure~\ref{fig:shodan-windows-78}. Adding
\verb|os:"Windows XP"| to the original search term
returned 2,174 results. Adding \verb|2003| to the original search term returned
35,560 results. Adding \verb|2008| to the original search term returned 36,802
results. Our custom scanning script, on the other hand, discovered 8,796 internet
facing devices on port 3389 after running a zmap~\cite{zmap} scan for 10000 seconds.
Then, after each layer of filtering, we had 8,796 IPs that we could connect to and
732 vulnerable devices out of those. We found considerably more vulnerable devices
with our custom tool for this CVE, likely due to the fact that there are just 
more devices with this vulnerability overall. Again, a comparison is given in tabular
form in Table~\ref{tab:cve-2019} with the vulnerability ratio listed at the bottom.

\begin{figure}[!h]
\begin{center}
\psfig{figure=images/port-3389-w78.eps,width=2in,height=3in} 
\caption{Shodan search results for CVE-2019-0708, where the device OS is Windows 7 or 8}
\label{fig:shodan-windows-78}
\end{center}
\end{figure}

\begin{table}[!h]
\begin{tabular}{r|l|l}
    \multicolumn{1}{l|}{}                                                                                         & Shodan    & Custom Tool \\ \hline
    \multicolumn{1}{l|}{Scan on Port 3389}                                                                        & 4,977,976 & 8,796       \\ \hline
    \multicolumn{1}{l|}{\begin{tabular}[c]{@{}l@{}}Vulnerable Devices Detected\\ (by OS for Shodan)\end{tabular}} &           &             \\ \hline
    Windows 7 or 8                                                                                                 & 14,012    &             \\ \hline
    Windows XP                                                                                                     & 2,174     &             \\ \hline
    Windows Server 2003                                                                                            & 35,560    &             \\ \hline
    Windows Server 2008                                                                                            & 36,802    &             \\ \hline
    Total                                                                                                          & 88,548    & 732         \\ \hline
    \multicolumn{1}{l|}{Vulnerable to Detected Ratio}                                                             & 0.01779   & 0.08322     \\
\end{tabular}
\caption{Comparison of Shodan results and our custom tool's results for CVE-2019-0708.}
\label{tab:cve-2019}
\end{table}

Finally, for CVE-2018-0101, the Shodan search for\\
\verb|"Set-Cookie: webvpn=;" ssl:"ASA Temporary"| returned\\42,246results.
The top five countries these results were from are United States, United Kingdom,
Germany, Canada, and Italy, with the exact numbers shown in
Figure~\ref{fig:shodan-cisco}. A more generic search for \verb|port:"443"| gave
42,311,357 results, but includes devices used for purposes besides a VPN server.
Our custom tool found 9,016IPs after scanning for 10000 seconds, but found that,
out of those, 104 had "ASA Temporary" as the common name in their SSL Certificates.
A side by side comparison of the number of results from each tool is shown in
Table~\ref{tab:cve-2018}.

\begin{figure}[!h]
\begin{center}
\psfig{figure=images/cisco.eps,width=2in,height=3in} 
\caption{Shodan search results for CVE-2018-0101}
\label{fig:shodan-cisco}
\end{center}
\end{figure}

\begin{table}[!h]
\begin{tabular}{l|l|l}
                             & Shodan     & Custom Tool \\ \hline
Scan on Port 443             & 40,311,357 & 9,016       \\ \hline
Vulnerable Devices Detected  & 42,246     & 104         \\ \hline
Vulnerable to Detected Ratio & 0.00104799 & 0.011535    \\
\end{tabular}
\caption{Comparison of Shodan results and our custom tool's results for CVE-2018-0101.}
\label{tab:cve-2018}
\end{table}
