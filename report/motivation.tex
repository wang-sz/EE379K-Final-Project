%--------------------------------------------
% \begin{figure}[t]
% \begin{center}
% \vspace{-0.2in}
% %%\psfig{figure=1bcounter.eps,width=2.5in,height=1.8in} 
% \psfig{figure=1bcounter.eps,width=1.7in,height=1.2in} 
% \caption{A 1-bit counter with reset. With the conventional technique of OR-ing all input shadow values, the feedback loop ensures that a 
% counter shall never be trusted once it gets marked as untrusted. Our shadow logic is more precise and recognizes that a trusted reset 
% guarantees a trusted $0$ in the counter value.}
% \label{fig:1bcounter}
% \end{center}
% \end{figure}
%--------------------------------------------

% -describe what real world concern or issues does the project aim to solve, why you feel this was a good project that people should care about.
% -Why having a tool to scan the internet helpful/useful
% -rename section to: (eg, Ideology, Practical implications etc.) 
% - Think from an internal security compliance team perspective -- you want to make sure all your apps are upto the latest security standard so that a breach is not easy
% -Think from pen-test engineers/consultancies -- they want a nice and effective way to detect vulnerabilities in systems so that the client's purpose is solved and they gain reputation etc.

The number of devices connected to the Internet are growing exponentially, increasing
the chances that malicious actors will be able to find vulnerable Internet-connected devices.
These devices can expose a wide variety of remotely accessible services
and are easily identified through search engines, like Shodan, designed for IoT devices.
This poses as a privacy concern for many as Shodan's port scans can reveal a lot of
private information about a device. We created a tool to scan the Internet
because it is useful in finding the vulnerabilities in a device that malicious actors
would use to gain access to any device, or, for instance, any device that
I may have connected to a network, for instance. Then the user is able to 
take precautionary security steps to prevent any cybercriminal hacking on his or her
device, such as modifying firewall rules or restricting Internet access.
Thus, by performing regular scans on the Internet, a user is able to keep up to date
if any new devices are being connected to the network. 