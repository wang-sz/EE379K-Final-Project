%--------------------------------------------
% \begin{figure}[t]
% \begin{center}
% \vspace{-0.2in}
% %%\psfig{figure=1bcounter.eps,width=2.5in,height=1.8in} 
% \psfig{figure=1bcounter.eps,width=1.7in,height=1.2in} 
% \caption{A 1-bit counter with reset. With the conventional technique of OR-ing all input shadow values, the feedback loop ensures that a 
% counter shall never be trusted once it gets marked as untrusted. Our shadow logic is more precise and recognizes that a trusted reset 
% guarantees a trusted $0$ in the counter value.}
% \label{fig:1bcounter}
% \end{center}
% \end{figure}
%--------------------------------------------

For this project, the works that we will build on most are the "Evaluation of
the ability of the Shodan search engine to identify Internet-facing industrial
control devices", "Impact of the Shodan Computer Search Engine on Internet-facing
Industrial Control System Devices," and "Contactless Vulnerability Analysis using
Google and Shodan."

All of the above mentioned works cover the Shodan search engine which will be
helpful in starting this project and knowing what we can do to perform our own
form of vulnerability analysis. The second report explains in more detail
compared to the other two about the Shodan program itself, such as its
functionality and device identification, indexing, as well as its setup and
deployment. Together, these reports in addition to previous labs involving
internet scanning will help us achieve our goal in discovering vulnerable
machines found by our script.

The CVEs listed previously were chosen due to their commonality, since this
would mean that there would be more information to work with. Additionally,
that gives the custom scanner more opportunities in the situation that it
does not perform as well as Shodan, so that a better comparison can be made
between the two scanning methods.
