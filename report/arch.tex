Since there are three different CVEs to look at, the methods used to scan
for devices vulnerable to those will be slightly different, both with
Shodan and our custom scanning tool. The CVEs listed below were chosen due 
to their commonality, since this would mean that there would be more 
information to work with. Additionally, that gives the custom scanner more 
opportunities in the situation that it does not perform as well as Shodan, 
so that a better comparison can be made between the two scanning methods.

% https://nvd.nist.gov/vuln/detail/CVE-2014-2256
% https://globalresilience.northeastern.edu/2018/02/using-shodan-as-a-tool-to-find-vulnerable-devices/
% https://stackoverflow.com/questions/47652705/find-device-information-from-ip-address
For CVE-2014-2256, the vulnerability is described as an issue with Siemens
SIMATIC S7-1200 CPU PLC devices that have an older firmware that allows remote
attackers to execute a Denial of Service (DoS) attack\cite{CVE-2014-2256}. On Shodan, the exact
search term used to find these vulnerable devices is: \verb|siemens port:"102"|.
The DoS attack is caused by attackers sending ISO-TSAP, or TCP, packets to
specifically port 102, since this port is left open by the firmware. Then, for
the custom tool, we use \verb|zmap|~\cite{zmap} to scan on port 102 and get a list of devices
probed. Then, we filter for devices by establishing a socket connection for each probed IP
and grabbing a banner from the device to get more information. We then search the information
for "Siemens" to filter the list of IPs for vulnerable devices.

% https://nvd.nist.gov/vuln/detail/CVE-2019-0708
% https://socprime.com/en/blog/proactive-detection-content-cve-2019-0708-vs-attck-sigma-elastic-and-arcsight/
CVE-2019-0708 is described as a remote code execution vulnerability that
exists in Microsoft's Remote Desktop Protocol in Windows\cite{CVE-2019-0708}. In Shodan, we need
several searches for devices with port 3389 open and running versions of 
Windows that could potentially be vulnerable. This means our searches combine
the term \verb|port:"3389"| with one of the following: \verb|os:"Windows 7 or 8"|,
\verb|os:"Windows XP"|, \verb|2003|, or \verb|2008|. In this case, the last two
terms refer to versions of Windows Server. Our custom tool for this vulnerability follows
a similar pipeline to the previous vulnerability on Siemens devices, utilizing \verb|zmap|~\cite{zmap}
to gather IPs probed on port 3389. However, in this case, after connecting to each IP and
attempting to get a banner from each one, we search the banner information for any of the
following terms: \verb|Windows 7|, \verb|Windows 8|, \verb|Windows XP|, or \verb|Windows Server|.

% https://nvd.nist.gov/vuln/detail/CVE-2018-0101
% https://info.stack8.com/blog/numerating-cisco-asa-systems-affected-by-cve-2018-0101-using-shodan
% https://cymmetria.com/blog/honeypot-cisco-asa-vulnerability/
% https://www.exploit-db.com/exploits/43986
Finally, CVE-2018-0101 is a vulnerability in Cisco's Adaptive Security Appliance (ASA)
Software, specifically the Secure Sockets Layer (SSL) VPN functionality. This
vulnerability could allow remote execution of a code and is possible since
enabling the webvpn feature will cause an attempted double free on a region
of memory on the device. This is detectable by checking to see if the Cisco
ASA device has the webvpn feature enabled. The National Vulnerability Database
provides a list of specific devices affected by this vulnerability\cite{CVE-2018-0101}. In Shodan,
we use the search term \verb|"Set-Cookie: webvpn=;" ssl:"ASA Temporary"|,
since the ASA Software sets the webvpn cookie. Using just the webvpn filter
may also capture other devices where some other web applications might have
set the cookie as well, but the ASA Temporary SSL certificate filter gives
a narrower set of search results of devices vulnerable to this CVE. 
Our custom tool to scan for this vulnerability has a similar pipeline, probing for
IPs on port 443 instead. Additionally, rather than grab a banner from the device,
we want to look for the term "ASA Temporary" in the SSL Certificate. Using the Python
module for OpenSSL~\ref{openssl}, we attempted to obtain SSL Certificates from each IP
that was probed and for the successful attempts, we then search the certificate
for the term.
