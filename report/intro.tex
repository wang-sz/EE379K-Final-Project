% First paragraph on the technology/society trends that lead 
% to the problem at hand.
With the growth and popularization of the Internet of Things (IoT),
more and more devices are becoming connected to each other through the Internet. 
This allows many devices to now have the convenience and accessibility of being 
connected to other devices through the Internet, such as printing to a
shared printer connected to the Internet or controlling security cameras
remotely.

% Second para: describe the key problem that if solved would make
% an impact. Why the current approaches leave a gap?
However, a key problem that comes hand in hand with these advantages is that
any vulnerabilities across millions of endpoints could allow unwanted and
unauthorized guests to control the device remotely. Devices without proper
security in place are vulnerable to attacks such as Denial of Service or a remote
control protocol. Attackers can abuse these gaps in security to gain access to
information the victim might have or hijack the device and use it for their own
purposes. If a device on a subnet were left unsecured and gets hijacked, attackers
could even use it as a backdoor to other devices connected to the same subnet.
There are plenty of tools that attackers can use to discover potentially useful
information to perform these attacks, such as zmap~\cite{zmap}, nmap~\cite{nmap},
and Shodan~\cite{shodan}.
Both zmap~\cite{zmap} and nmap~\cite{nmap} are used to scan the Internet, but the
latter can give much more detailed information, such as the device's name, OS, 
and even hardware information through OS fingerprinting and banner grabbing. 
However, the search engine tool, Shodan, can
provide a way for attackers to filter IPs for certain vulnerabilities as well and
it works by searching and indexing any connected device. These devices range from home routers
to traffic lights, and Shodan can produce useful information through it's banner
grabbing capabilities, similar to nmap. The service banner
contains various information for an IoT device, including geographic location, 
IP address, software version, make and model, etc. for specific ports. 
Currently, there are many computer systems, including traffic lights, security cameras,
or industrial control systems, that have little to no security, leaving
them vulnerable to attackers\cite{afit//CSAR-10-025-01}.

% Third: describe your approach. Key insight that enables your approach,
% and what is novel/interesting about the insight.
The approach taken in this paper is to determine some Common Vulnerabilities
and Exposures (CVEs) that can be detected remotely and then utilize Shodan
to see how many machines have those vulnerabilities and what similarities
those machines have. Shodan is a quick and easy way to explore the large
spanned IoT and detect key vulnerabilities in Internet-facing devices.
Although many people believe Shodan is a privacy concern due to its detailed
device-level insights, it is completely legal and serves to only collect
already available data to the public. After scanning the ports through Shodan,
our next step is creating a custom scanning tool that will be used to see if we obtain
similar results to that of Shodan's scan. 

% Fourth, fifth: Delve deeper into the approach and experimental setup. 
% In the final report, describe key findings.
Overall, the setup is fairly straightforward.
The data returned by Shodan's search is split up into several pieces based
on the different search terms used and by the CVE that the search terms
target. On the other hand, the methods for a custom scanner for specific
CVEs depend on the specific CVEs chosen, but have a similar pipeline:
scan for IPs that respond on a specific port, establish socket connections
with those IPs, grab banners from devices we can connect to, and search
the banners for information of interest. In this case, we look at
CVE-2014-2256, CVE-2019-0708, and CVE-2018-0101.

Evaluating the performance of our custom tool, we see that although our
results are not as numerous and impressive as the Shodan searches, it is
still possible to find these vulnerabilities in some devices with relative
ease.

% End with outline or what comes next and why. 
The main contributions of this paper are as follows:
\begin{itemize}
    \item We explore more about the vulnerabilities being searched for and
        provide more detail on how these can be exploited for some specific
        systems.
    \item We show how these vulnerabilities can be detected using Shodan
        and what information can be gathered from these scans. We analyze
        this information and discuss how this could be used maliciously.
    \item We create our own scanner and compare its results to Shodan to
        get insight on how information of specific systems is exposed to
        the Internet.
\end{itemize}
