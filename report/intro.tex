% First paragraph on the technology/society trends that lead 
% to the problem at hand.
With the growth and popularization of the Internet of Things (IoT),
more and more devices are becoming connected to each other through the Internet. 
This allows many devices to now have the convenience and accessibility of being 
connected to other devices through the Internet, such as printing to a
shared printer connected to the Internet or controlling security cameras
remotely.

% Second para: describe the key problem that if solved would make
% an impact. Why the current approaches leave a gap?
However, a key problem that comes hand in hand with these advantages is that
any vulnerabilities across millions of endpoints could allow unwanted and
unauthorized guests to control the device remotely. Devices without proper
security can be attacked and, thus, the attackers are able to control the 
devices and gain access to any information the victim might have. In some cases,
attackers can utilize the connection and control they have over the device to then 
attack the other devices connected to the same network. The search engine tool, Shodan, can
be used by attackers to help them search for Internet-connected devices. Shodan works by
searching and indexing any connected device, ranging from webcams to traffic lights, and can produce
useful information through Shodan's banner grabbing capabilities. The service banner
contains various information for an IoT device, including geographic location, 
IP address, softare version, make and model, etc. for specific ports. 
Currently, there are many computer systems, including traffic lights, security cameras,
or industrial control systems, that have little to no security, leaving them vulnerable to attackers\cite{afit//CSAR-10-025-01}.

% Third: describe your approach. Key insight that enables your approach,
% and what is novel/interesting about the insight.
The approach taken in this paper is to determine some Common Vulnerabilities
and Exposures (CVEs) that can be detected remotely and then utilize Shodan
to see how many machines have those vulnerabilities and what similarities
those machines have. Shodan is a quick and easy way to explore the large
spanned IoT and detect key vulnerabilities in Internet-facing devices.
Although many people believe Shodan is a privacy concern due to its detailed
device-level insights, it is completely legal and serves to only collect
already available data to the public. After scanning the ports through Shodan,
our next step is creating a custom scanning tool that will be used to see if we obtain
similar results to that of Shodan's scan. 

% Fourth, fifth: Delve deeper into the approach and experimental setup. 
% In the final report, describe key findings.
Overall, the setup is simple and is mostly data analysis and comparison.
The data returned by Shodan's search will be divided by the CVE that the
machine is vulnerable to, but will also need to have the flexibility to
pick a machine and see what CVEs Shodan determined it to be vulnerable to.

The methods for a custom scanner for specific CVEs will depend on the
specific CVEs chosen. In this case, we plan to look at CVE-2014-2256,
CVE-2019-0708, and CVE-2018-0101.

% End with outline or what comes next and why. 
The main contributions of this paper are as follows:
\begin{itemize}
    \item We explore more about the vulernabilities being searched for and
        provide more detail on how these can be exploited for some specific
        systems.
    \item We show how these vulernabilities can be detected using Shodan
        and what information can be gathered from these scans. We analyze
        this information and discuss how this could be used maliciously.
    \item We create our own scanner and compare its results to Shodan to
        get insight on how information of specific systems is exposed to
        the internet.
\end{itemize}
