% First paragraph on the technology/society trends that lead 
% to the problem at hand.
With the growth and popularization of the intnert more and more devices
are becoming connected to each other through the internet. This allows many
devices to now have the convenience and accessibility of being connected
to other devices through the internet, such as printing to a
shared printer connected to the internet or controlling security cameras
remotely.

% Second para: describe the key problem that if solved would make
% an impact. Why the current approaches leave a gap?
However, a key problem that comes hand in hand with these advantages is that
any any vulnerabilities could allow unwanted and unauthorized guests to
control the device remotely. Devices without proper security can be attacked
and the attackers can then control the devices and gain access to any
information the device might have. In some cases, attackers can utilize the
connection and control they have over the device to then attack the other
devices connected to the same network. For example, Shodan is a tool that
allows users to search for devices connected to the internet and can give
potentially useful information for attackers. Currently, there are many
computer systems, including traffic lights, security camers, or industrial
control systems, that have little to no security, leaving them vulnerable to
attackers\cite{afit//CSAR-10-025-01}.

% Third: describe your approach. Key insight that enables your approach,
% and what is novel/interesting about the insight.
The approach taken in this paper is to determine some Common Vulnerabilities
and Exposures (CVEs) that can be detected remotely and then utilize Shodan
to see how many machines have those vulnerabilities and what similarities
the machines have. Then, since one approach to defense against this type of
detection is to just prevent detection by Shodan, a custom scanning tool
will be used to see if it has similar results.

% Fourth, fifth: Delve deeper into the approach and experimental setup. 
% In the final report, describe key findings.
Overall, the setup is simple and is mostly data analysis and comparison.
The data returned by Shodan's search will be divided by the CVE that the
machine is vulnerable to, but will also need to have the flexibility to
pick a machine and see what CVEs Shodan determined it to be vulnerable to.

The methods for a custom scanner for specific CVEs will depend on the
specific CVEs chosen. In this case, we plan to look at CVE-2014-2256,
CVE-2019-0708, and CVE-2018-0101.

% End with outline or what comes next and why. 
The main contributions of this paper are as follows:
\begin{itemize}
    \item We explore more about the vulernabilities being searched for and
        provide more detail on how these can be exploited for some specific
        systems.
    \item We show how these vulernabilities can be detected using Shodan
        and what information can be gathered from these scans. We analyze
        this information and discuss how this could be used maliciously.
    \item We create our own scanner and compare its results to Shodan to
        get insight on how information of specific systems is exposed to
        the internet.
\end{itemize}
