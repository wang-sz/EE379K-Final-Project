
% Point out other important approaches in the problem area. For example, if you 
% are proposing an architecture, maybe OS or PL approaches to this problem. 

% The following paragraph included just for a figure. The caption of a figure is very
% important -- I try to tell the entire story in the figures and captions alone, 
% just in case that is all the reader sees.

% The general problem of determining whether information flows in a program from
% variable $x$ to variable $y$ is undecidable, as ``any procedure purported to
% decide it could be applied to the statement {\bf if}~$f(x)$~halts {\bf then}~$y
% := 0$ and thus provide a solution to the halting problem for arbitrary
% recursive function''~\cite{denning-impossible}.  

Bodenheim \cite{afit//CSAR-10-025-01} discusses the metrics
on whether or not Shodan is being used to target industrial control
system devices (ICS), and addresses the concern of 
whether this method and channel of attack are widely used when targetting 
a specific set of devices. The research done in this paper evaluates Shodan's impact 
on Internet-connected ICS device security by using a series of Internet-facing ICS honeypots.
These honeypots were designed to be representative of ICS devices that are 
currently found through Shodan. Thus, allowing him to analyze network activity by measuring
transmission control protocol (TCP) connections, total number of TCP packets, and
the number of distinct IP addresses interacting with each honeypot. Overall, they found
Shodan does not impact Internet-facing ICS device security, but is identified as a passive
reconnaissance tool.

In another paper, Bodenheim et. al.\cite{bodenheim_shodan_ics},
analyzes Shodan's detection ability on a specific programmable logic controller (PLC)
and suggests a potential solution to mitigate its visibility to Shodan. Four PLCs were
configured to be Internet-connecting and deployed to evaulate Shodan's indexing and
querying proficiencies. All the PLCs were exposed to two ports, Port 80 and Port 44818,
while two of these were designed to have altered service banners in order to prevent
Shodan query discovery. The results show that Shodan was able to successfully 
index and identify all the deployed PLCs within 19 days, thus showing the
expansive capabilities of Shodan.

Ercolani et. al.\cite{shodan_vis} discusses how Shodan actually
scans the Internet. Shodan works by scanning the Internet for open ports on 
IP addresses and determines runnings services on chosen ports. In order to
visualize these open ports, they create visualizations that use both IP addresses
and open ports as nodes. The team focuses on identifying Supervisory Control and
Data Acquisition (SCADA) and Industrial Control System (ICS) devices due to the
infrastructural support of these devices. These Shodan visualizations pose a
potential solution in understanding what devices are running on a network.


% In "Contactless Vulnerability Analysis using Google and Shodan"\cite{google_shodan},
%the paper discusses combining Google searches and Shodan searches to determine
%the vulnerability of systems in large scale networks. This contactless vulnerability
%analysis suggests a potential solution in analyzing vulnerable domains by refining search terms and 
%combining results from Google and Shodan.